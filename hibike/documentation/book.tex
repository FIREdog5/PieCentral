\documentclass[12pt]{book}
\usepackage{tgschola}
\usepackage[margin=1in]{geometry}
\usepackage{minted}

\title{The Hibike and Runtime Guide}
\date{\today}
\author{Brose Johnstone}
\begin{document}
\maketitle

\tableofcontents

\chapter{Introduction}
The driving principle behind PiE is running a robotics competition. This means many things:
logistics, provinding mentorship and guidance, organizing events,
providing an interface for programming the robot, and making
the robot work. Keep in mind that this guide addresses only one aspect
of the competition; plumbing is not everything. With that said, it is still a good
idea to have a plumber on hand when pipes start to leak.
\section{A Brief Overview of the Control Stack}
For our purposes, there are four elements involved in controlling the robot.
\begin{enumerate}
\item Dawn: the main interface that students use. Takes inputs from the controller,
and shows data and errors to students.
\item Runtime: the program in charge of monitoring and communication. Sends and receives data from
Dawn, as well as monitoring Hibike and StateManager.
\item StateManager: a data store. Provides a central place to store and retrive data from Hibike
and Runtime.
\item Hibike: the program that communicates with sensors. Handles low-level protocol details.
\end{enumerate}
\section{An Analogy}
A simple way to explain the role of each component in the stack, from Dawn to Hibike, is
as an analogy to a company.

Dawn is the CEO, the main person in charge. Every other system starts or stops
on her word, and follows her orders to the letter. Runtime is the middle manager. He
passes information from underlings on to the boss, relays her orders to them, and makes
sure that they are working. StateManager is the record-keeper. He stores information
that Hibike and Runtime give him, and retrieves it when they ask for it. Hibike is
in charge of a bunch of interns, the sensors. She asks them to periodically give her
updates on what they are doing, and fires them if they don't.
\chapter{Hibike}
Hibike is in charge of low-level sensor communication, and all the gory
details that are involved. It is divided into two modules:
\begin{itemize}
\item \begin{verbatim}hibike_message.py\end{verbatim}
The low-level details of the
Hibike communications protocol. Handles things like encoding into packets and checksums.
\item \begin{verbatim}hibike_process.py\end{verbatim}
The ``supervisor'' of sensors.
Communicates with sensors, sends data to runtime, and reacts to orders from StateManager.
\end{itemize}

Alongside these main modules, there are various testing modules:
\begin{itemize}
\item \begin{verbatim}virtual_device.py\end{verbatim}
Virtual devices, for testing Hibike's read and write capabilities.
\item \begin{verbatim}hibike_tester.py\end{verbatim}
A test module that wraps a Hibike process with a nicer interface.
\end{itemize}

\section{The Hibike Protocol}
\section{The Hibike Process}
\section{Testing}
\chapter{StateManager}
\chapter{Runtime}
\end{document}